\chapter{Introduction}

Games like chess, shogi, go, and checkers, are games in which each player has perfect information regarding the state of the game. According to \cite{MYCIELSKI199241}, perfect information means that there is no hidden information between the players, each time only one of the player moves, that the game depends only on their choices, every player remembers the past, and in principle they know all the possible futures of the game.

Throughout the history, these games have been benchmarks for intelligence. That's why numerous attempts have been made to beat humans in these games. According to \cite{hsu2002behind}, DeepBlue is the first artificial intelligence to defeat a grandsmaster in the game of chess in a professional setting.

\section{Background of the Study}

Artificial intelligence (AI) has made significant strides in recent years, achieving groundbreaking results in areas previously thought to require human intuition and expertise. Notable among these advancements is AlphaZero, an AI developed by DeepMind that has demonstrated an ability to master complex games such as Chess, Go, and Shogi. By employing deep reinforcement learning and self-play, AlphaZero has set new standards in artificial intelligence by learning game strategies autonomously without any human input beyond the basic rules of each game. This breakthrough has inspired researchers to explore applications of AlphaZero’s learning model in diverse domains.

One such potential application is in the game of Damath. Damath is a Filipino educational board game that combines the principles of mathematics with the strategic nature of a board game. The game is similar to Checkers but integrates mathematical operations into each move, requiring players to use mathematical strategies along with spatial and strategic thinking. As a tool for enhancing students' mathematical and analytical skills, Damath has become popular in Philippine schools. Applying AlphaZero to Damath presents an opportunity not only to advance AI applications in local games but also to explore new approaches to educational AI, contributing to both the fields of AI research and educational technology.

\section{Problem Statement}

Current implementations of AI in educational board games are often limited in their capacity to mimic human intuition and strategic depth. Most Damath AIs rely on pre-programmed rules or decision trees that do not fully explore the strategic potential of the game. This limitation results in AI opponents that are relatively easy to predict and exploit, reducing the educational value and engagement level for players. An AlphaZero-based model could provide a self-learning AI capable of adapting to the unique strategies and mathematical complexities of Damath, resulting in a more challenging and dynamic opponent.

This research proposes to explore the effectiveness of AlphaZero in developing an AI agent for Damath, investigating whether the self-play reinforcement learning approach can produce an AI capable of challenging human players at various skill levels. Additionally, the study will assess the educational potential of a stronger, more versatile Damath AI in supporting the development of players’ mathematical and strategic skills.

\section{Objectives of the Study}

The primary objective of this study is to develop and evaluate an AlphaZero-based AI model tailored to the game of Damath. Specific objectives include:
\begin{enumerate}
    \item To design a self-learning AI agent for Damath based on the AlphaZero architecture, implementing appropriate adjustments for the mathematical rules unique to the game.
    \item To evaluate the performance of the AlphaZero-based Damath AI by testing it against both human players and traditional rule-based Damath AIs.
    \item To analyze the AI’s learning process and behavior, examining whether it can effectively learn and adapt to optimal Damath strategies without external guidance.
    \item To assess the educational impact of the AlphaZero-based Damath AI on players, particularly in terms of enhancing their mathematical and strategic thinking abilities.
\end{enumerate}

\section{Significance of the Study}

The significance of this study lies in its potential contributions to both artificial intelligence research and educational gaming. From a technical perspective, adapting AlphaZero to a game like Damath requires innovations in handling mathematical operations as part of the AI’s decision-making process, a feature not present in typical board games like Chess or Go. Success in this endeavor could provide insights into the adaptability of AlphaZero in various educational and non-standard game environments, offering a model for future AI research in educational games.

From an educational perspective, this study could improve the effectiveness of Damath as a teaching tool, offering students a more challenging and interactive experience. An advanced AI opponent could encourage students to think critically, refine their strategies, and improve their mathematical skills, contributing to a deeper understanding and appreciation of mathematics through play.

\section{Scope and Limnitations}

This study will focus on adapting the AlphaZero framework specifically to the rules and gameplay of Damath. The AI will be trained to play the standard version of the game, assuming knowledge of basic mathematical operations—addition, subtraction, multiplication, and division—as they apply to Damath moves. The study will not cover alternative or modified versions of Damath. Additionally, given computational constraints, the AI’s training process may be limited to a defined number of games or iterations.

Limitations of the study include computational resources, as AlphaZero requires significant processing power for optimal training results. Additionally, given the relatively small player base for Damath compared to globally popular games like Chess or Go, gathering sufficient human players for evaluation purposes may pose a challenge.

\section{Organization of the Study}

The succeeding chapters of this study are organized as follows:
\begin{itemize}
    \item Chapter 2: Preliminaries – Discusses previous works and studies on AlphaZero, Damath, and the application of AI in educational games.
    \item Chapter 3: Methodology – Details the approach, model design, and implementation process for developing the AlphaZero-based Damath AI.
    \item Chapter 4: Results and Discussion – Presents and analyzes the results of the AI’s performance and its educational implications.
    \item Chapter 5: Conclusion and Recommendations – Summarizes the study’s findings and suggests areas for future research.
\end{itemize}

In the following sections, we delve deeper into the research questions and review the current state of AI applications in educational games, setting the foundation for this investigation into AlphaZero’s applicability to Damath.